\documentclass[a4paper,12pt]{article}
\usepackage[latin1]{inputenc}
\usepackage[T1]{fontenc}
\usepackage[finnish]{babel}

\begin{document}
\section*{Curriculum vitae}

\subsection*{Henkil�tiedot}
Mikko Nummelin\\
%P�iv�nkaari 14 E\\
%02210, Espoo\\
gsm: +358-50-5406205\\
email: mikko.nummelin\_AT\_tkk.fi\\
\\
Syntym�vuosi: 1975\\

\subsection*{Koulutus}

\begin{itemize}
\item
Jatko-opiskelijana Teknillisess� Korkeakoulussa 2008\ -\\
Tutkimusala: Teknillinen matematiikka (erityisesti numeeriset konformikuvaukset) \\
Syvent�v� aihealue: Ohjelmistoj�rjestelm�t
\item
{\bf Diplomi-insin��ri}: \\
2007 (Teknillinen korkeakoulu) \\
P��aine: Teknillinen matematiikka \\
Sivuaine: Ohjelmistoj�rjestelm�t
\item
Ylioppilas: \\
1994 (Etel�-Tapiolan lukio)
\end{itemize}

\subsection*{Kielitaidot}

\begin{itemize}
\item
suomi (�idinkieli)
\item
englanti (kohtalaisen hyvin, peruskoulussa ja lukiossa A-englanti,
olen asunut 1981-1982 ja 1988 Yhdysvalloissa yhteens� 11 kk.)
\item
viro (t�ll� tulee toimeen, olen k�ynyt 2 puolen vuoden kurssia Espoon
ty�v�enopistolla.)
\item ruotsi (t�ll� tulee toimeen, peruskoulussa ja lukiossa B-ruotsi,
lis�ksi virkamiesruotsi suoritettu Teknillisess� korkeakoulussa.)
\item
saksa (jonkin verran, peruskoulussa ja lukiossa C-saksa)
\end{itemize}

\newpage
\subsection*{Ty�historia ja nykytilanne}

\begin{itemize}
\item
7.1.2008 - nyky��n:\\
Ixonos Oyj,\\
ohjelmistosuunnittelija.
\item
syksy 2007:\\
Tutkimusapulainen ja assistentti Teknillisen korkeakoulun matematiikan laitoksella.
\item
Diplomity� aiheesta:
"Konformikuvausten konstruoiminen yhdesti yhten�isilt� kompleksitason
alueilta kanonisille alueille", tehty kes�ll� 2007 Teknillisen
korkeakoulun matematiikan laitokselle
\item
kev�t 2005-kev�t 2007:\\
Osa-aikaisesti tuntiassistentin teht�vi�
Teknillisen korkeakoulun matematiikan laitoksella
\item
kev�t 2002-syksy 2004:\\
Affecto Oy (loppuvaiheessa AffectoGenimap),\\
tekninen tukikonsultti
\item
kes� 2001:\\
Regex Oy,\\
ohjelmistosuunnittelija
\item
elokuu 2000:\\
Espoon liiketalousinstituutti,\\
mikrotukihenkil�
\item
1998-2001:\\
Osa-aikaisesti tuntiassistentin teht�vi�
Teknillisen korkeakoulun matematiikan laitoksella
\item
1997-1998:\\
Opti Inter-Consult (Bentley Finland),\\
ohjelmistosuunnittelija
\end{itemize}

\subsection*{Ohjelmointikielet}

C, C++, Java, HTML, SQL, (Z80 ja i386-assembler, Scheme, Prolog, Tcl/Tk, Ruby, Perl)

\subsection*{Matemaattiset ohjelmistot}

Mathematica, Matlab/Octave, Mathcad, Maple, Yacas, (Maxima)
\end{document}
